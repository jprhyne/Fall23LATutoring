% Hacky way of allowing us to compile this with and without answers using the makefile in this folder
\ifdefined\isanswers
    \documentclass[answers]{exam}
\else
    \documentclass{exam}
\fi
% Use mystyle file. This is a congolmeration of things I've used over the years. Most is stolen from either previous professors or stack overflow. It may or may not be cited. Assume I didn't come up with any of it myself.
\usepackage{mystyle}

\title{Bin 1 Problems}
\date{\today}
\author{}

\begin{document}
\maketitle
This document holds problems that fit into the Bin 1 according to the prelim syllabus. 

\tableofcontents

\section{Prelim Problems}
The problems are listed in most recent first. IE the first question is from August 2023, second is from January 2023, etc. In addition, the proofs provided are my own to complement the ones posted by the committees of previous years.
\begin{questions}
    \question Let $T$ be a linear map $T:U\rightarrow V$ and $S$ be a linear map $S:V\rightarrow W$.\\
    Prove that $\dim U-\dim V \leq \dim\nullS ST - \dim\nullS S$.
    \begin{solution}\,\\
        \underline{\textbf{Solution with intuition building sprinkled in}}\\
        This problem uses the rank nullity theorem (Axler calls this the fundamental theorem of linear maps).

        We can get this intuition from the fact that we are being asked to relate the dimension of spaces to the dimension of the range/null space of a linear map FROM these spaces.

        Reminder of the rank nullity theorem:
        \begin{theorem}\label{thm:RN}
            If $U,V$ are vector spaces, and $T\in\mathcal{L}\left(U,V\right)$, then
            \begin{equation}\label{eq:RN}
                \mathsf{dim}\, U = \mathsf{dim}\,\range T + \mathsf{dim}\,\mathsf{null}\, T
            \end{equation}
        \end{theorem}

        Well, this would introduce a dimension based on ranges. But we'll hope it goes away through some algebra or hiding it behind the $\leq$ operator.

        Since we want to have $\dim\nullS ST$ and $\dim\nullS S$ present, we should try to use theorem \ref{thm:RN} in a way that would give us these operators on the right side.

        See that because $T$ maps from $U$ to $V$ and $S$ maps from $V$ to $W$, then $ST$ maps from $U$ to $W$. This means by using theorem \ref{thm:RN}, we get
        \[
            \dim U = \dim\nullS ST + \dim\range ST
        \]
        Next, we can use theorem \ref{thm:RN} to get that
        \[
            \dim V = \dim\nullS S + \dim\range S
        \]
        By subtracting these two equations, we have
        \[
            \dim U - \dim V = \dim\nullS ST + \dim\range ST - \left(\dim\nullS S + \dim\range S\right)
        \]
        So if we group together our range and nullspace dimensions we are close to what we need
        \[
            \dim U - \dim V = \dim\nullS ST - \dim\nullS S + \left(\dim\range ST - \dim\range S\right)
        \]
        So, we are really close to what we want. All that is left is just try try to get rid of our range dimensions through some kind of algebra. Since we are wanting a $\leq$ in our final statement, if  we can argue that 
        $ \left(\dim\range ST - \dim\range S\right) \leq 0 $ then we would be done.

        So, we are essentially trying to argue $\dim\range ST \leq \dim\range S$. 

        Before talking about how we would approach this, recall that $\range ST$ is the set of all
        vectors in $W$ that $S(T(u))$ can produce for any $u\in U$, while $\range S$ is the set of all
        vectors in $W$ that $S(v)$ can produce for any $v\in V$. 

        We also know from theorem 3.19 from Axler that $\range T$ is a subspace of $V$. This means that if we look 
        at $\range ST$ we are looking at what $S$ can map to from $\range T$. So this means we are mapping
        a potentially smaller (but never larger) set inside $V$ to $W$ meaning that $\range ST \leq \range S$

        So we can conclude that
        \begin{align*}
            \dim U - \dim V &= \dim\nullS ST - \dim\nullS S + \left(\dim\range ST - \dim\range S\right) \\
            &\leq \dim\nullS ST - \dim\nullS S + \left(\dim\range S - \dim\range S\right) \\
            &= \dim\nullS ST - \dim\nullS S
        \end{align*}
        as desired.
        \\\underline{\textbf{Proof written up properly for the prelim}}\\
        From the definition of $S$ and $T$ in the problem description, we know that $ST:U\rightarrow W$. From the rank-nullity theorem, we know the following
        \begin{align*}
            \dim U &= \dim\nullS ST + \dim\range ST \\
            \dim V &= \dim\nullS S + \dim\range S
        \end{align*}
        We also know from Axler that $\range T$ is a subspace of $V$ thus $\range ST$ is a subspace of $\range S$
        because we are mapping a subset of $V$ to $W$ with $ST$. This allows us to say that
        \[
            \dim\range ST \leq \dim\range S
        \]

        Combining all of the above results we have
        \begin{align*}
            \dim U - \dim V &= \dim\nullS ST + \dim\range ST - \left(\dim\nullS S + \dim\range S\right) \\
            &= \dim\nullS ST - \dim\nullS S + \left(\dim\range ST - \dim\range S\right) \\
            &\leq \dim\nullS ST - \dim\nullS S + \left(\dim\range S - \dim\range S\right) \\
            &= \dim\nullS ST - \dim\nullS S
        \end{align*}

        which is our desired result
    \end{solution}
    \question Suppose $U,W$ are subspaces of a finite-dimensional vector space $V$.
    \begin{parts}
        \part Show that $\dim \left(U\cap W\right) = \dim U + \dim W - \dim\left(U + W\right)$
        \part Let $n=\dim V$. Show that if $k<n$, then an intersection of $k$ subspaces of dimension $n-1$ always has dimension at least $n-k$.
    \end{parts}
    \question Prelim Aug 22 Part 1.
    \begin{parts}
        \part Let $v_1,v_2,v_3$ be linearly dependent, and $v_2,v_3,v_4$ are linearly independent
        \begin{subparts}
            \subpart Show that $v_1$ is a linear combination of $v_2$ and $v_3$.
            \subpart Show that $v_4$ is not a linear combination of $v_1,v_2,$ and $v_3$.
        \end{subparts}
        \iffalse % This is not really a good question for study purposes. Nothing really to gain from this other than practicing showing something is a basis, and there are more straightforward examples for that. But I am leaving this in the source file in case someone would like to see it.
        \part Find $10$ vectors in $\R^3$ so that any three of them for a basis. Justify your answer.
        \fi
    \end{parts}
    \question Let $V$ be a vector space of dimension $n$ over a field $F$. Let $v_1,v_2,\dots,v_n$ be a basis of $V$ and $T$ be an operator on $V$. \\
    Prove: $T$ is invertible if and only if $Tv_1,Tv_2,\dots,Tv_n$ are linearly independent.
\end{questions}

\section{Axler Problems}
\begin{questions}
    \question Suppose
    \[  
        U = \left\{\left(x,y,x+y,x-y,2x\right)\in\F^5:x,y\in\F\right\}.
    \]
    Find a subspace $W$ of $\F^5$ such that $\F^5 = U\oplus W.$

    \question Suppose
    \[  
        U = \left\{\left(x,y,x+y,x-y,2x\right)\in\F^5:x,y\in\F\right\}.
    \]
    Find three subspaces $W_1,W_2,W_3$ of $\F^5$ such that $\F^5 = U\oplus W_1\oplus W_2\oplus W_3.$

    \question Prove or give a counter example: If $U_1,U_2,W$ are subspaces of $V$ such that
    \[
        V = U_1\oplus W \text{ and } V = U_2\oplus W
    \]
    Then $U_1 = U_2$.
\end{questions}

\section{Other Problems}
\end{document}
