% Hacky way of allowing us to compile this with and without answers using the makefile in this folder
\ifdefined\isanswers
    \documentclass[answers]{exam}
\else
    \documentclass{exam}
\fi
% Use mystyle file. This is a congolmeration of things I've used over the years. Most is stolen from either previous professors or stack overflow. It may or may not be cited. Assume I didn't come up with any of it myself.
\usepackage{mystyle}
\renewcommand{\L}[1]{\mathcal{L}\left(#1\right)}
\newcommand{\ip}[1]{\left\langle#1\right\rangle}
\newcommand{\M}[1]{\mathcal{M}\left(#1\right)}
\newcommand{\latex}{\LaTeX}
\newcommand{\spanS}[1]{\text{Span}\left(#1\right)}

\title{Bin 2 Problems}
\date{\today}
\author{}

\begin{document}
\maketitle
This document holds problems that fit into the Bin 2 according to the prelim syllabus. Or are approached 
in a way most compatible with Bin 2

\tableofcontents

\section{Prelim Problems}
\begin{questions}
    \question Let $u$ be a unit vector in an $n-\text{dimensional}$ inner product space $V$ over $\R$. Define 
    $T\in\L{V}$ as:
    \[
        T(x) = x - 2\ip{x,u}u, \quad x\in V
    \]
    Show that 
    \begin{parts}
        \part $T$ is an isometry
        \begin{solution}

            \secName{Intuition} For this problem, we will need a couple of facts. The first one is that
            we want to extend $u$ to be an orthonormal basis of $V$. the reason this would be a good place to
            start is because we see an inner product in the definition of $T$ and also the definition of an 
            isometry is as follows
            \begin{defin} Axler 7.37\\
                An operator $S\in\L{V}$ is called an isometry if for all $v\in V$, 
                \[
                    \norm{Sv}{} = \norm{v}{}
                \]
            \end{defin}
            Which is an inner product squared. So if we have an orthonormal basis, these equalities are much much
            easier to deal with.

            Now that we have $(u,u_2,\dots,u_n)$ is an orthonormal basis of $V$, then we see that 
            \begin{align*}
                T(u) &= u - 2\ip{u,u}u = -u \\
                T(u_k) &= u_k - 2\ip{u_k,u}u = u_k \qquad\qquad \text{for }k=2,\dots,n
            \end{align*}

            At this point, it is possible to be done. If we happen to remember Theorem 7.42 in Axler, we can 
            appeal to parts (a) and (d) on that one and that's it. For this problem, even though this was in the second half, it has
            $3$ parts so we are not trivializing this problem nor making it exceptionally short. However, if this
            problem was in the first half and it only had 1 or 2 parts, it would be a good idea to not appeal
            to this theorem. If you are unsure, ask the proctor or err on the side of caution

            \begin{theorem} Axler 7.42
                Suppose $S\in\L{V}$. Then the following are equivalent.
                \begin{enumerate}
                    \item $S$ is an isometry
                    \item $\ip{Su,Sv} = \ip{u,v}$\qquad\qquad for all $u,v\in V$
                    \item $Se_1,\dots,Se_n$ is orthonormal for every orthonormal list of vectors
                        $e_1,\dots,e_n$ in $V$
                    \item There exists some orthonormal basis $e_1,\dots,e_n$ of $V$ such that 
                        $Se_1,\dots,Se_n$ is orthonormal
                    \item $S^*S = I$
                    \item $SS* = I$
                    \item $S^*$ is an isometry 
                    \item $S$ is invertible and $S^{-1} = S^*$
                \end{enumerate}
            \end{theorem}

            However, let's say we forget this theorem or we don't feel comfortable appealing to it. Then we would
            need to show that our operator satsifies the previous definition directly. For ease of notation, we
            assign $u_1 = u$ in our previous orthonormal basis.

            So, let $v\in V$, we aim to show that $\norm{Tv}{} = \norm{v}{}$

            Useful theorems for this part are Axler 6.25 and 6.30. These are listed below for convenience

            \begin{theorem} Axler 6.25 \label{thm:625}

                If $e_1,\dots,e_n$ is an orthonormal list of vectors in $V$, then 
                \[
                    \norm{\sum_{k=1}^na_ke_k}{}^2 = \sum_{k=1}^n |a_k|^2
                \]
            \end{theorem} 

            \begin{theorem} Axler 6.30 \label{thm:630}

                Suppose $e_1,\dots,e_n$ is an orthonormal basis of $V$ and $v\in V$. Then
                \begin{align}
                    v &= \sum_{k=1}^n \ip{v,e_k}e_k \nonumber\\
                    \norm{v}{}^2 &= \sum_{k=1}^n\left|\ip{v,e_k}\right|^2\label{eq:normV}
                \end{align}
            \end{theorem}

            So we will show that $\norm{Tv}{}^2$ equals equation \eqref{eq:normV}. In order to do this,
            we will first write out what $Tv$ is in terms of the coefficients of $v$. See that
            \begin{align*}
                Tv &= T\left(\sum_{k=1}^n \ip{v,u_k}u_k\right) \\
                &= \sum_{k=1}^n \ip{v,u_k}Tu_k \\
                &= -\ip{v,u}u + \sum_{k=2}^n \ip{v,u_k}u_k
            \end{align*}

            Notice that this is almost the exact same as $v$! In fact, it only has the first term negated. We are
            also in a form compatible with Theorem \ref{thm:625}, so we have that 
            \begin{align*}
                \norm{Tv}{}^2 &= \norm{-\ip{v,u}u + \sum_{k=2}^n \ip{v,u_k}u_k}{}^2 \\
                &= |-\ip{v,u}|^2 + \sum_{k=2}^n |\ip{v,u_k}|^2 \\
                &= |\ip{v,u}|^2 + \sum_{k=2}^n |\ip{v,u_k}|^2 \\
                &= \sum_{k=1}^n |\ip{v,u_k}|^2 \\
                &= \norm{v}{}^2
            \end{align*}
            Now, take the square root of both sides, and we get that
            \[
                \norm{Tv}{} = \norm{v}{}
            \]
            Since $v$ was an arbitrary vector in $V$, this holds for all vectors in $V$. Thus $T$ is an isometry
            as desired.

            \secName{Solution}

            \textbf{\underline{Aside}:} For this one, I am going to show directly that 
            $\norm{Tv}{} = \norm{v}{}$ for all $v\in V$.

            First, extend $u$ to an orthonormal basis of $V$ denoted $u_1,\dots,u_n$ where $u_1 = u$. See that
            for these basis vectors, 
            \begin{align*}
                T(u) &= u - 2\ip{u,u}u = -u \\
                T(u_k) &= u_k - 2\ip{u_k,u}u = u_k \qquad\qquad \text{for }k=2,\dots,n
            \end{align*}

            Now, we aim to show that $\norm{Tv}{} = \norm{v}{}$ for all $v\in V$.

            Let $v\in V$. We know from Theorem 6.30 in Axler that
            \begin{align*}
                v &= \sum_{k=1}^n \ip{v,u_k}u_k \\
                \norm{v}{}^2 &= \sum_{k=1}^n\left|\ip{v,u_k}\right|^2
            \end{align*}

            This means that we have from above
            \begin{align*}
                Tv &= T\left(\sum_{k=1}^n \ip{v,u_k}u_k\right) \\
                &= \sum_{k=1}^n \ip{v,u_k}Tu_k \\
                &= \ip{v,u}Tu + \sum_{k=2}^n \ip{v,u_k}Tu_k \\
                &= -\ip{v,u}u + \sum_{k=2}^n \ip{v,u_k}u_k
            \end{align*}

            From Theorem 6.25, we can now say that

            \begin{align*}
                \norm{Tv}{}^2 &= \norm{-\ip{v,u}u + \sum_{k=2}^n \ip{v,u_k}u_k}{}^2 \\
                &= |-\ip{v,u}|^2 + \sum_{k=2}^n |\ip{v,u_k}|^2 \\
                &= |\ip{v,u}|^2 + \sum_{k=2}^n |\ip{v,u_k}|^2 \\
                &= \sum_{k=1}^n |\ip{v,u_k}|^2 \\
                &= \norm{v}{}^2
            \end{align*}
            Where the last step is the second equation in Theorem 6.30.

            Next, take the square root of both sides, and we get that
            \[
                \norm{Tv}{} = \norm{v}{}
            \]
            Since $v$ was an arbitrary vector in $V$, this holds for all vectors in $V$. Thus $T$ is an isometry
            as desired.

        \end{solution}
        \part If $A=\M{T}$ is a matrix representation of $T$, then $\det A = -1$ 
        \begin{solution}
            
            \secName{Note} There are two ways to approach this problem. The solution provided by the prelim 
            committee takes a matrix algebra approach. I will briefly justify why they made the claim that they
            do, but I will take a different approach for this one.

            \secName{Intuition}
            
            \secName{Matrix Algebra Approach}
            
            The committee argues that $\M{T}$ with respect to our above basis is the matrix
            \[
                \M{T} = \begin{bmatrix}
                    -1 & 0 \\
                    0  & I_{n-1}
                \end{bmatrix}
            \]

            This is a consequence of our equations where we apply $T$ to $u$ and $u_2,\dots,u_n$. And then the
            determinant of this matrix is $-1*1^{n-1} = -1$. And since we have that the determinant of a matrix
            is invariant under similarity transformations (change of basis matrices and a standard result in 
            many undergraduate linear algebra courses) we have that the determinant of $\M{T}$ under any 
            basis is $-1$ as desired.

            \secName{Axler-esque Approach}

            In Axler the determinant of an operator $T$ is defined as follows
            \begin{defin}
                Suppose $T\in\L{V}$.
                \begin{itemize}
                    \item If $\F = \C$, then the determinant of $T$ is the product of the eigenvalues of $T$,
                        with each eigenvalue repeated according to its multiplicity
                    \item If $\F = \R$, then the determinant of $T$ is the product of the eigenvalues of $T_\C$,
                        with each eigenvalue repeated according to its multiplicity
                \end{itemize}
            \end{defin}
            Where $T_\C$ is the complexification of $T$. Essentially this just means that we allow complex 
            eigenvalues if our operator happened to have them. (But this isn't relevant here, as we will 
            show shortly)

            We showed in the previous part that $V$ has an orthonormal basis consisting of eigenvectors of $T$. 
            $u,u_2,\dots,u_n$ are eigenvectors where $u$ is associated with eigenvalue $-1$ and 
            $u_2,\dots,u_n$ are associated with the eigenvalue $1$. This means that $T$ is diagonalizable 
            according to the Real Spectral Theorem. The eigenspace associated with $-1$ has dimension $1$ 
            while the eigenspace associated with $1$ has dimension $n-1$ as we have $n-1$ linearly independent
            vectors living in this space. This means from our above definition,
            \[
                \det T = -1*1^{n-1} = -1
            \]
            Since the determinant is a property of the operator and not the matrix representation of the 
            operator (according to Axler and formalized with Theorem 10.42) we know that for any matrix 
            representation of $T$, denoted $A$,
            we have that $\det A = \det T = -1$ as desired.
            
            \secName{Solution}

            (I am only writing up the Axler-esque approach, but the above Matrix Algebra Approach would be 
            sufficient if this approach made more sense to you.)


            Since we have that $u,u_2,\dots,u_n$ is an orthonormal basis of $V$ and 
            \begin{align*}
                T(u) &= u - 2\ip{u,u}u = -u \\
                T(u_k) &= u_k - 2\ip{u_k,u}u = u_k \qquad\qquad \text{for }k=2,\dots,n
            \end{align*}
            we have that $u$ is an eigenvector of $T$ with associated eigenvalue of $-1$, and 
            $u_2,\dots,u_n$ are $n-1$ linearly independent eigenvectors of $T$ associated with the 
            eigenvalue $1$. Since the eigenspace associated with $-1$ is of dimension $1$ and the eigenspace
            associated with $1$ is of dimension $n-1$, we have that by the definition of the determinant 
            \[
                \det T = -1*1^{n-1} = -1
            \]
            
            So by Theorem 10.42, for any matrix representation of $T$, denoted $A$, is given by
            \[
                \det A = \det T = -1
            \]
            as desired.
        \end{solution}
        \part If $S\in\L{V}$ is an isometry with $1$ as an eigenvalue, and if the eigenspace of $1$ is of 
        dimension $n-1$, then there exists some $w\in V$ where $w$ is a unit vector and for all $x\in V$:
        \[
            S(x) = x - 2\ip{x,w}w
        \]
        \begin{solution}
            
            (Note: we are using $*$ to denote the adjoint here. Since we are in a real vector space, this is 
            equivalent to just the transpose).

            \secName{Intuition} We will probably want to follow a similar path as above since this would allow us
            to "reuse" some of our intuition that we have built up so far. So, we will first consider the 
            eigenspace associated with $1$. Let $v_1,\dots,v_{n-1}$ be an orthonormal basis of $E(1,S)$. Next, 
            we want to be able to use the fact that $S$ is an isometry. From Axler 7.42 (see above intuition 
            section) we know that since $S$ is an isometry, we have $SS^* = I$. So, we would need this to hold
            for any matrix representation. For matrix representations, we need to have a basis, so we extend
            $v_1,\dots,v_{n-1}$ to a basis of $V$ denoted $v_1,\dots, v_{n-1}, v_n$ (Note: Without Loss of 
            Generality, assume that $v_n$ is a unit vector). We will want to show that
            $v_n$ is an eigenvector associated with $-1$. 

            Let's write out what this matrix representation looks 
            like with respect to this basis. (Note: we don't know anything about $v_n$ yet!)

            \[
                A = \M{S,(v_1,\dots,v_n)} = \begin{bmatrix}
                    1 & 0 & \dots & a_{1n} \\
                    0 & 1 & \dots & a_{2n} \\
                    \vdots &\vdots&\ddots & \vdots \\
                    0 & 0 & \dots & a_{nn}
                \end{bmatrix}
            \]
            We know the first $n-1$ columns of $A$ are the identity matrix as the vectors $v_1,\dots,v_{n-1}$
            are eigenvectors associated with $1$. So, let's compute $SS^*$. (As this is a pain in \latex, I am
            going to do block multiplication for the first part. If anything is confusing, I will happily 
            clarify).

            See that $\M{S^*,(v_1,\dots,v_n)} = A^*$, so we need to just compute $AA^*$.

            Let $b$ be the first $n-1$ rows of the $n^\text{th}$ column of $A$ (This will make writing $A$ as a
            block matrix much easier in the first couple steps. We will use the actual entries of $A$ before
            we make any real computations though!).

            \begin{align*}
                I_n = AA^* &= \begin{bmatrix}
                    1 & 0 & \dots & a_{1n} \\
                    0 & 1 & \dots & a_{2n} \\
                    \vdots &\vdots&\ddots & \vdots \\
                    0 & 0 & \dots & a_{nn}
                \end{bmatrix}\begin{bmatrix}
                    1 & 0 & \dots & 0 \\
                    0 & 1 & \dots & 0 \\
                    \vdots & \vdots & \ddots & \vdots \\
                    a_{1n} & a_{2n} & \dots & a_{nn}
                \end{bmatrix} \\
                &= \begin{bmatrix}
                    I_{n-1} & b \\
                    \vec{0} & a_{nn}
                \end{bmatrix}\begin{bmatrix}
                    I_{n-1} & \vec{0} \\
                    {b}^\top & {a_{nn}}
                \end{bmatrix} \\
                &= \begin{bmatrix} I_{n-1} \\ \vec{0} \end{bmatrix} 
                   \begin{bmatrix} I_{n-1} & \vec{0} \end{bmatrix} + 
                   \begin{bmatrix} a_{1n}\\\vdots\\a_{nn} \end{bmatrix}
                       \begin{bmatrix} {a_{1n}}&\dots&{a_{nn}}\end{bmatrix}\\
                &= \begin{bmatrix} I_{n-1} & \vec{0} \\
                    \vec{0} & 0\end{bmatrix} + 
                    \begin{bmatrix}
                        a_{1n}{a_{1n}} & a_{1n}{a_{2n}} & \dots & a_{1n}{a_{nn}} \\
                        a_{2n}{a_{1n}} & a_{2n}{a_{2n}} & \dots & a_{2n}{a_{nn}} \\
                        \vdots & \vdots &\ddots & \vdots \\
                        a_{nn}{a_{1n}} & a_{nn}{a_{2n}} & \dots & a_{nn}{a_{nn}} 
                    \end{bmatrix} \\
                &= \begin{bmatrix} I_{n-1} & \vec{0} \\
                    \vec{0} & 0\end{bmatrix} + 
                    \begin{bmatrix}
                        {a_{1n}}^2 & a_{1n}{a_{2n}} & \dots & a_{1n}{a_{nn}} \\
                        a_{2n}{a_{1n}} & {a_{2n}}^2 & \dots & a_{2n}{a_{nn}} \\
                        \vdots & \vdots &\ddots & \vdots \\
                        a_{nn}{a_{1n}} & a_{nn}{a_{2n}} & \dots & {a_{nn}}^2 
                    \end{bmatrix}
            \end{align*}
            Now, the first matrix in our sum has $1's$ on the diagonal of the first $n-1$ columns and $0's$ 
            elsewhere. So in order for out final matrix to be the identity, we need the following $n$ equations to
            hold (we'll worry about off diagonals afterward for the second matrix in our sum).
            \begin{align*}
                {a_{1n}}^2 &= 0 \\
                &\vdots\\
                {a_{(n-1)n}}^2 &= 0 \\
                {a_{nn}}^2 &= 1
            \end{align*}
            The first $n-1$ equations give us that for $k=1,\dots,n-1$,
            \[
                a_{kn} = 0
            \]

            (Note: This means that the off diagonals of our second matrix in the above sum must be $0$ as needed)

            And the final equation gives us that
            \[
                a_{nn} = \pm 1
            \]

            However, we know that if $a_{nn} = 1$, then we would have that $v_n$ satisfies $Sv_n = v_n$, which
            is not possible because we would have $n$ linearly independent eigenvectors associated with $1$ which
            violates our assumption that $\dim E(S,1) = n-1$. So, we know $a_{nn} = -1$. 

            This means that for our basis vectors, we have
            \begin{align*}
                Sv_1 &= v_1 \\
                     &\vdots\\
                Sv_{n-1}&= v_{n-1}\\
                Sv_n &= -v_n
            \end{align*}
            Next, we show that if we assign $w=v_n$, then the desired equation satisfies the above behavior on 
            these basis vectors (which would allow us to conclude that $S(x) = x - 2\ip{x,v_n}v_n$ for all
            $x\in V$).


            See that for $k=1,\dots,n$, 
            \[
                v_k - 2\ip{v_k,v_n}v_n = v_k = S(v_k)
            \]
            The first equation is because from Axler 7.22 (as $S$ is diagonalizable, it is clearly Normal) 
            eigenvectors associated with distict eigenvalues are orthogonal. Finally see that

            \[
                v_n - 2\ip{v_n,v_n}v_n = v_n - 2v_n = -v_n = S(v_n)
            \]

            So since our equation holds for all basis vectors, it must also hold for all $x\in V$. Thus, we have
            shown that there exists $w\in V$ such that for all $x\in V$, $S(x) = x - 2\ip{x,w}w$ as desired.
            
            \secName{Solution}

            Let $(v_1,\dots,v_{n-1}$ denote an orthonormal basis of $E(S,1)$. We know that such a basis exists
            because we are assuming that $\dim E(S,1) = n-1$. Next, extend this basis to a basis of $V$. Denote
            this $v_1,\dots,v_{n-1},v_n$. Without loss of generality, assume that $\norm{v_n}{} = 1$. We will
            now show that $v_n$ is an eigenvector associated with eigenvalue $-1$.

            See that the matrix representation of $S$ with respect to this matrix is given by
            \[
                A = \M{S,(v_1,\dots,v_n)} = \begin{bmatrix}
                    1 & 0 & \dots & a_{1n} \\
                    0 & 1 & \dots & a_{2n} \\
                    \vdots &\vdots&\ddots & \vdots \\
                    0 & 0 & \dots & a_{nn}
                \end{bmatrix}
            \]
            We write the final column this way as we don't yet know how $v_n$ relates to the other vectors.
            Since $S$ is an isometry, Axler 7.42 gives us that $SS^* = I$ where $I$ is the 
            identity operator on $V$. So, we know that $AA^* = I_n$ where $I_n$ is the $n$-by-$n$ identity matrix.
            For ease of notation, let $b = \begin{bmatrix}a_{1n}&\dots&a_{(n-1)n}\end{bmatrix}^\top$

            This means
            \begin{align*}
                I_n = AA^* &= \begin{bmatrix}
                    1 & 0 & \dots & a_{1n} \\
                    0 & 1 & \dots & a_{2n} \\
                    \vdots &\vdots&\ddots & \vdots \\
                    0 & 0 & \dots & a_{nn}
                \end{bmatrix}\begin{bmatrix}
                    1 & 0 & \dots & 0 \\
                    0 & 1 & \dots & 0 \\
                    \vdots & \vdots & \ddots & \vdots \\
                    a_{1n} & a_{2n} & \dots & a_{nn}
                \end{bmatrix} \\
                &= \begin{bmatrix}
                    I_{n-1} & b \\
                    \vec{0} & a_{nn}
                \end{bmatrix}\begin{bmatrix}
                    I_{n-1} & \vec{0} \\
                    {b}^\top & {a_{nn}}
                \end{bmatrix} \\
                &= \begin{bmatrix} I_{n-1} \\ \vec{0} \end{bmatrix} 
                   \begin{bmatrix} I_{n-1} & \vec{0} \end{bmatrix} + 
                   \begin{bmatrix} a_{1n}\\\vdots\\a_{nn} \end{bmatrix}
                       \begin{bmatrix} {a_{1n}}&\dots&{a_{nn}}\end{bmatrix}\\
                &= \begin{bmatrix} I_{n-1} & \vec{0} \\
                    \vec{0} & 0\end{bmatrix} + 
                    \begin{bmatrix}
                        a_{1n}{a_{1n}} & a_{1n}{a_{2n}} & \dots & a_{1n}{a_{nn}} \\
                        a_{2n}{a_{1n}} & a_{2n}{a_{2n}} & \dots & a_{2n}{a_{nn}} \\
                        \vdots & \vdots &\ddots & \vdots \\
                        a_{nn}{a_{1n}} & a_{nn}{a_{2n}} & \dots & a_{nn}{a_{nn}} 
                    \end{bmatrix} \\
                &= \begin{bmatrix} I_{n-1} & \vec{0} \\
                    \vec{0} & 0\end{bmatrix} + 
                    \begin{bmatrix}
                        {a_{1n}}^2 & a_{1n}{a_{2n}} & \dots & a_{1n}{a_{nn}} \\
                        a_{2n}{a_{1n}} & {a_{2n}}^2 & \dots & a_{2n}{a_{nn}} \\
                        \vdots & \vdots &\ddots & \vdots \\
                        a_{nn}{a_{1n}} & a_{nn}{a_{2n}} & \dots & {a_{nn}}^2 
                    \end{bmatrix}
            \end{align*}

            Since we need these two matrices to sum to be $I_n$, we need the following $n$ equations to hold, 
            which we get from ensuring the diagonal elements are correct in the final sum.
            \begin{align*}
                {a_{1n}}^2 &= 0 \\
                &\vdots\\
                {a_{(n-1)n}}^2 &= 0 \\
                {a_{nn}}^2 &= 1
            \end{align*}
            The first $n-1$ equations give us that for $k=1,\dots,n-1$,
            \[
                a_{kn} = 0
            \]

            (Note: This means that the off diagonals of our second matrix in the above sum must be $0$ as needed)

            And the final equation gives us that
            \[
                a_{nn} = \pm 1
            \]

            However, we know that if $a_{nn} = 1$, then we would have that $v_n$ satisfies $Sv_n = v_n$, which
            is not possible because we would have $n$ linearly independent eigenvectors associated with $1$ which
            violates our assumption that $\dim E(S,1) = n-1$. So, we know $a_{nn} = -1$. 

            Next, see that
            \begin{align*}
                Sv_1 &= v_1 \\
                     &\vdots\\
                Sv_{n-1}&= v_{n-1}\\
                Sv_n &= -v_n
            \end{align*}
            Next, we show that if we assign $w=v_n$, then the desired equation satisfies the above behavior on 
            these basis vectors (which would allow us to conclude that $S(x) = x - 2\ip{x,v_n}v_n$ for all
            $x\in V$).


            See that for $k=1,\dots,n$, 
            \[
                v_k - 2\ip{v_k,v_n}v_n = v_k = S(v_k)
            \]
            The first equation is because from Axler 7.22 (as $S$ is diagonalizable, it is clearly Normal) 
            eigenvectors associated with distict eigenvalues are orthogonal. Finally see that

            \[
                v_n - 2\ip{v_n,v_n}v_n = v_n - 2v_n = -v_n = S(v_n)
            \]

            So since our equation holds for all basis vectors, it must also hold for all $x\in V$. Thus, we have
            shown that there exists $w\in V$ such that $w$ is a unit vector and  
            for all $x\in V$, $S(x) = x - 2\ip{x,w}w$ as desired.

        \end{solution}
    \end{parts}
    \question Let $V$ be a finite dimensional real vector space with basis $e_1,\dots,e_n$ (the standard basis of
    $\R^n$)
    \begin{parts}
        \part Let $A$ be a positive definite bijective matrix in $V$ (This means $A$ is a matrix representation 
        of some invertible linear operator in $\L{V}$). For any $v,w\in V$, expressed as coordinate vectors according to 
        this basis (their standard
        representation if you were to write them down), define 
        \[
            \ip{v,w} := v^\top Aw.
        \]
        Show that this is an inner product.
        \begin{solution}
            
            \secName{Intuition}
            This is a computational part. Just need to show the following properties are satisfied
            \begin{defin} Inner Product Axler 6.3
                An operation $\ip{\cdot,\cdot}:V\times V \rightarrow \F$ is an inner product if the
                following are satisfied.
                \begin{enumerate}
                    \item Positivity\\
                        For all $v\in V$, $\ip{v,v} \geq 0$
                    \item Definiteness\\
                        $\ip{v,v} = 0 \iff v=0$
                    \item Additivity in the first slot\\
                        for all $u,v,w\in V$, we have $\ip{u+v,w} = \ip{u,w} + \ip{v,w}$
                    \item Homogeneity in the first slot\\
                        For all $\lambda\in\F, u,v\in V$, we have $\ip{\lambda u,v} = \lambda\ip{u,v}$
                    \item Conjugate symmetry\\
                        For all $u,v\in V$, we have $\ip{u,v} = \overline{\ip{v,u}}$
                \end{enumerate}
            \end{defin}
            However, since we are in a real space, we can show this with only $\F=\R$

            How we would do this is just directly using the given properties of $A$

            \secName{Solution}

            First, we show positivity. Let $v\in V$. See that 
            \begin{align*}
                \ip{v,v} &= v^\top Av \\
                &\geq 0
            \end{align*}
            Since $A$ is positive.

            Next, we show definiteness. Assume that $v=0$. See that
            \[
                \ip{v,v} = v^\top Av = \vec{0}^\top A\vec{0} = 0
            \]
            Now, let $v\in V$ such that $\ip{v,v} = 0$. This means
            \[
                0 = v^\top Av
            \]
            However, since $A$ is positive definite, $v$ must be $0$. Therefore, $\ip{v,v} = 0 \iff v=0$ as 
            desired.

            Next, we show additivity in the first slot. Let $u,v,w\in V$. See that:
            \begin{align*}
                \ip{u + v,w} &= (u + v)^\top Aw \\
                &= (u^\top + v^\top )Aw \\
                &= u^\top Aw + v^\top Aw \\
                &= \ip{u,w} + \ip{v,w}
            \end{align*}
            As desired.

            Next, we show homogeneity in the first slot. Let $\lambda\in\R$ and $u,v\in V$. See that
            \begin{align*}
                \ip{\lambda u,v} &= (\lambda u)^\top Av \\
                &= \lambda u^\top Av \\
                &= \lambda \ip{u,v} 
            \end{align*}
            As desired.

            Finally, we show conjugate symmetry (Since we are in $\R$, we instead show symmetry). 
            Let $u,v\in V$. See that
            \begin{align*}
                \ip{u,v} &= u^\top Av \\
                &= (u^\top Av)^\top \qquad\qquad \text{Since this is a scalar}\\
                &= v^\top A^\top u \\
                &= v^\top Au\qquad\qquad \text{Since $A$ is symmetric due to being positive}\\
                &= \ip{v,u}
            \end{align*}

            Thus, we have shown that this function is an inner product as desired.
        \end{solution}
        \part Let $\ip{\cdot,\cdot}$ be an inner product in $V$. Define $A$ to be a matrix such that 
        $A_{ij} = \ip{e_i,e_j}$ is a positive bijective matrix such that $\ip{v,w} = v^\top Aw$.
        \begin{solution}

            The idea behind this problem is that if we compute the inner product of all possible combinations 
            of basis vectors, and store them in a matrix (Where the index of the ``first'' basis vector gives
            the row and the index of the ``second'' gives the column). Then for any pair of vectors in our vector
            space we can instead compute the above mentioned matrix vector product. As far as usefulness, this
            could potentially be useful in cases where computing the inner product is expensive to do (or in 
            an even more ideal situation, cheap for the basis vectors) and we want to compute a lot of inner 
            products.

            This problem has two steps. First is to show that $A$ is positive definite and invertible and the
            second is to show that if we define $A$ to be this way, then we know that $\ip{v,w}=v^\top Aw$.
            This essentially means that we can write an inner product in this space as the matrix vector 
            multiplication instead.

            \secName{Intuition}
            Assuming that we show $\ip{v,w} = v^\top Aw$, then the needed properties of $A$ (positive definite
            and bijective) are nearly immediate consequences. So let $v,w\in V$, See that in the problem statement
            we write these vectors as coordinates in our giving basis. This means
            \begin{align*}
                v &= \sum_{k=1}^n v_ke_k \\
                w &= \sum_{k=1}^n w_ke_k
            \end{align*}
            So, we have that 
            \begin{align*}
                \ip{v,w} &= \ip{\sum_{i=1}^n v_ie_i,\sum_{j=1}^n w_je_j} \\
                &= \sum_{i=1}^n\ip{v_ie_i,\sum_{j=1}^n w_je_j} \\
                &= \sum_{i=1}^n\sum_{j=1}^n\ip{v_ie_i,w_je_j} \\
                &= \sum_{i=1}^n\sum_{j=1}^nv_iw_j\ip{e_i,e_j} \\
                &= \sum_{i=1}^n\sum_{j=1}^nv_iw_jA_{ij}
            \end{align*}
            Next we will show that $v^\top Aw$ looks the same.

            \begin{align*}
                v^\top Aw &= v^\top \begin{bmatrix}
                    \sum_{j=1}^nA_{1j}w_j \\
                    \vdots \\
                    \sum_{j=1}^nA_{nj}w_j 
                \end{bmatrix} \\
                &= \begin{bmatrix} v_1 & \dots & v_n \end{bmatrix}\begin{bmatrix}
                    \sum_{j=1}^nA_{1j}w_j \\
                    \vdots \\
                    \sum_{j=1}^nA_{nj}w_j 
                \end{bmatrix} \\
                &= \sum_{i=1}^n v_i\sum_{j=1}^n A_{ij}w_j \\
                &= \sum_{i=1}^n \sum_{j=1}^n v_iw_j A_{ij}
            \end{align*}

            Which is the same as $\ip{v,w}$. Thus we have that for all $v,w\in V$, $\ip{v,w} = v^\top Aw$.

            Now all that is left to show is that $A$ is positive definite and invertible. For positive, let $v\in V$. Then, $v^\top Av = \ip{v,v} \geq 0$.

            For definiteness, let $v\in V$ such that $v^\top Av = 0$. See that
            \[
                0 = v^\top Av = \ip{v,v} \iff v = \vec{0}.
            \]
            Thus, $A$ is positive definite. Now for invertability. We will do this by showing the nullspace
            is trivial. IE let $v\in V$ such that $Av = \vec{0}$. This tells us
            \[
                Av = \vec{0} \iff v^\top Av = v^\top\vec{0} = 0
            \]
            However recall that $v^\top Av = \ip{v,v}$ which is only $0$ when $v = \vec{0}$. Thus, the only
            vector in the nullspace of $A$ is the zero vector. Thus, it is invertible as $V$ is a finite vector
            space.

            \secName{Solution}
            First, we will show that for all $v,w\in V$, we have that $\ip{v,w} = v^\top Aw$. We will do this 
            by expanding both sides to the same form, which will let us conclude that $\ip{v,w}=v^\top Aw$.
            See first that 
            \begin{align*}
                \ip{v,w} &= \ip{\sum_{i=1}^n v_ie_i,\sum_{j=1}^n w_je_j} \\
                &= \sum_{i=1}^n\ip{v_ie_i,\sum_{j=1}^n w_je_j} \\
                &= \sum_{i=1}^n\sum_{j=1}^n\ip{v_ie_i,w_je_j} \\
                &= \sum_{i=1}^n\sum_{j=1}^nv_iw_j\ip{e_i,e_j} \\
                &= \sum_{i=1}^n\sum_{j=1}^nv_iw_jA_{ij}
            \end{align*}
            Next see:
            \begin{align*}
                v^\top Aw &= v^\top \begin{bmatrix}
                    \sum_{j=1}^nA_{1j}w_j \\
                    \vdots \\
                    \sum_{j=1}^nA_{nj}w_j 
                \end{bmatrix} \\
                &= \begin{bmatrix} v_1 & \dots & v_n \end{bmatrix}\begin{bmatrix}
                    \sum_{j=1}^nA_{1j}w_j \\
                    \vdots \\
                    \sum_{j=1}^nA_{nj}w_j 
                \end{bmatrix} \\
                &= \sum_{i=1}^n v_i\sum_{j=1}^n A_{ij}w_j \\
                &= \sum_{i=1}^n \sum_{j=1}^n v_iw_j A_{ij}
            \end{align*}
            Which is the same as $\ip{v,w}$. Thus we have that for all $v,w\in V$, $\ip{v,w} = v^\top Aw$. Now,
            we need only show that $A$ is positive definite and invertible. First positivity.

            Let $v\in V$. We have, by above, 
            \[
                v^\top Av = \ip{v,v} \geq 0
            \]
            Next, definiteness. Let $v\in V$ such that $v^\top Av = 0$. See that
            \[
                0 = v^\top Av = \ip{v,v} \iff v = \vec{0}.
            \]
            We also show the other direction. Let $v=\vec{0}$, we have
            \[
                v^\top Av = \vec{0}^\top A\vec{0} = 0
            \]

            Finally we show that $A$ is invertible. We will do this by showing the nullspace
            is trivial. IE let $v\in V$ such that $Av = \vec{0}$. This tells us
            \[
                Av = \vec{0} \iff v^\top Av = v^\top\vec{0} = 0
            \]
            However recall that $v^\top Av = \ip{v,v}$ which is only $0$ when $v = \vec{0}$. Thus, the only
            vector in the nullspace of $A$ is the zero vector. Thus, it is invertible as $V$ is a finite vector
            space.

        \end{solution}
    \end{parts}
    \question $V$ be a finite-dimensional inner product space over $\C$. Let $T$ be a normal operator
    on $V$. Let $\lambda\in\C$ and let $v\in V$ be a unit vector (ie $\norm{v}{} = 1$). Prove that $T$ has an
    eigenvalue $\lambda'$ such that 
    \[
        \norm{\lambda - \lambda'}{} \leq \norm{Tv - \lambda v}{}.
    \]
    \begin{solution}
        
        \secName{Intuition} Since we know nothing about $T$ other than it is normal. So we use the one thing
        that has not failed us yet (eigenvector basis!). 

        \secName{Solution}

        Since $T$ is normal, the complex spectral theorem
        gives us that there exists an orthonormal basis consisting of eigenvectors of $T$. Denote this basis
        $(v_1,\dots,v_n)$ with associated eigenvectors $\lambda_1,\dots,\lambda_n$.

        Next since we have a basis, we know that there exist some $a_1,\dots,a_n\in\C$ such that
        \[
            v = \sum_{k=1}^n a_kv_k
        \]
        Finally, we see that
        \begin{align*}
            \norm{Tv - \lambda v}{}^2 &= \norm{T\left(\sum_{k=1}^n a_kv_k\right) - \lambda\left(\sum_{k=1}^n a_kv_k\right)}{}^2 \\
            &= \norm{\sum_{k=1}^n a_kTv_k - a_k\lambda v_k}{}^2 \\
            &= \norm{\sum_{k=1}^n (\lambda_k - \lambda)a_kv_k}{}^2 \\
            &= \sum_{k=1}^n\abs{\lambda_k -\lambda}^2\abs{a_k}^2 \qquad\text{Axler 6.25} \\
            &\geq \sum_{k=1}^n \min_{\ell\in\{1,\dots,n\}}\abs{\lambda_\ell - \lambda}^2\abs{a_k}^2 \\
            &= \min_{\ell\in\{1,\dots,n\}}\abs{\lambda_\ell - \lambda}^2\sum_{k=1}^n\abs{a_k}^2\\
            &= \min_{\ell\in\{1,\dots,n\}}\abs{\lambda_\ell - \lambda}^2\norm{v}{}^2\qquad\text{Axler 6.25} \\
            &= \min_{\ell\in\{1,\dots,n\}}\abs{\lambda_\ell - \lambda}^2\qquad\text{$v$ is a unit vector} \\
            &= \abs{\lambda_j - \lambda}^2.\\
            &= \norm{\lambda_j - \lambda}{}^2
        \end{align*}
        The $j$ in the final equation is given by $j = \arg\min_{\ell\in\{1,\dots,n\}}\abs{\lambda_\ell - \lambda}^2$. 
        Finally, we redo some variable names and take the square root of both sides to get it in the form of 
        the problem statement.

        Define $\lambda' = \lambda_j$ and we finally have that
        \[
            \norm{\lambda - \lambda'}{} \leq \norm{Tv - \lambda v}{}
        \]
        as desired.

    \end{solution}
\end{questions}

\section{Axler Problems}
\begin{questions}
    \question Suppose $\ip{\cdot,\cdot}_1$ and $\ip{\cdot,\cdot}_2$ are inner products on $V$ over a field
    $\F$ such that for all 
    $u,v\in V$, $\ip{u,v}_1 = 0 \iff \ip{u,v}_2=0$. Prove that there is a positive number $c$ such that 
    $\ip{u,v}_1 = c\ip{u,v}$ for all $u,v\in V$.
    \begin{solution}

        \secName{Preamble}

        This problem is a bit difficult to do, and while not exactly in the syllabus of the Prelim (which is 
        why we didn't talk about it in our sessions) but seeing how a linear function can be used may be 
        informative and give some experience using unfamiliar concepts. This should be used as a guideline for
        how a proof would look (with explainations too). 
        So, we will skip the intuition section and instead only do the more proper solution
        writeup.

        This proof is a slightly modified version of the one provided by Axler. (want to be clear about credit 
        for it).

        \secName{Solution}

        Let $w\in V$ such that $w\neq\vec{0}$. Define the linear functionals 
        $\phi\,:\,V\rightarrow\F$ and $\psi\,:\,V\rightarrow\F$ as for all $v\in V$:
        \begin{align*}
            \phi(v) &= \ip{v,w}_1 \\
            \psi(v) &= \ip{v,w}_2
        \end{align*}
        From the problem statement, we know that $\nullS{\phi} = \nullS{\psi}$. Recall that $\phi$ and $\psi$
        are elements of the dual space $V'$. So we can talk about their spans as they are vectors themselves.
        Next, we also know that:
        \[
            \spanS{\phi} = \left(\nullS{\phi}\right)^0 = \left(\nullS{\psi}\right)^0 = \spanS{psi}
        \]
        This equation comes from the fundamental theorem of linear algebra (or linear maps) along with the 
        definition of the annihilator of a space. Since the spans of these functionals are the same, and they
        are singular vectors, we have that there exists some $c_w\in\F$ such that 
        \[
            \phi = c_w\psi
        \]
        Note: the subscript $w$ denotes the potential reliance on the chosen $w$.
        Now, we will show that $c_w\in\R$ and $c_w > 0$. Since above holds for all $v\in V$, we know it also works for $w$. See
        \[
            \phi(w) = c_w\psi(w) \implies \ip{w,w}_1 = c_w\ip{w,w}_2 \implies \norm{w}{1}^2 = c_w\norm{w}{2}^2.
        \]
        Where $\norm{\cdot}{1}$ denotes the norm induced by the inner product $\ip{\cdot,\cdot}_1$ and 
        $\norm{\cdot}{2}$ denotes the norm induced by the inner product $\ip{\cdot,\cdot}_2$. Since these
        are norms and $w\neq0$, the left hand side of our final equation is a real positive number, and 
        the norm on the right side is $c_w$ multiplied by a real non-negative number. Thus, we have that $c_w$
        must be both real and positive.

        Next, we show that $c_w$ is independent of our choice of $w$ so we can rename it to be $c$.
        Let $w,x\in V$ and $c_w,c_x\in\R$ such that for all $v\in V$ we have:
        \begin{align}
            \ip{v,w}_1 &= c_w\ip{v,w}_2 \label{eq:w}\\
            \ip{v,x}_1 &= c_x\ip{v,x}_2 \label{eq:x}
        \end{align}
        We will now show that $c_w = c_x$. Since the above works for all $v\in V$, plug in $x$ into \eqref{eq:w}
        and $w$ into \eqref{eq:x}. See that this gives us both
        \begin{align*}
            \ip{x,w}_1 &= c_w\ip{x,w}_2 \\
            \ip{w,x}_1 &= c_x\ip{w,x}_2
        \end{align*}
        We can use the conjugate symmetry property of $\ip{\cdot,\cdot}_1$ to get that
        \[
            \overline{\ip{x,w}}_1 = \ip{w,x}_1 = c_x\ip{w,x}_2
        \]
        which we can take the conjugate of the far left and far right sides to give us
        \[
            \ip{x,w}_1 = \overline{c_x\ip{w,x}}_2 = c_x\overline{\ip{w,x}}_2
        \]
        Putting all this together lets us simplify to
        \begin{align*}
            c_w\ip{x,w}_2 &= \ip{x,w}_1 \\
            &= c_x\overline{\ip{w,x}}_2 \\
            &= c_x\ip{x,w}_2
        \end{align*}

        So, we have that $c_w\ip{x,w}_2 = c_x\ip{x,w}_2$. Since the inner product is a function, we have that
        $c_w = c_x$. So the choice of $c_w$ is independent of the choice of vector in our previous linear 
        functional, so we drop the subscript and are left with a $c\in\R$ such that $c>0$ and for all $u,v\in V$
        we have
        \[
            \ip{u,v}_1 = c\ip{u,v}_2
        \]
        as desired.
    \end{solution}
    \question Let $V$ be an inner product space over a field $\F$.
    Suppose $e_1,\dots,e_m$ is an orthonormal list of vectors in $V$. Let $v\in V$. Prove that
    \[
        \norm{v}{}^2 = |\ip{v,e_1}|^2 + \cdots + |\ip{v,e_m}|^2
    \]
    if and only if $v\in\text{span}\left({e_1,\dots,e_m}\right)$.
    \begin{solution}

        \secName{Intuition} 
        This is asking us to prove a little stronger version of Axler 6.25 and 6.30. While the
        wording is slightly different, it is similar in at least the backwards direction. Due to this, the 
        backwards direction is a little easier to do, so we start there. (Also, we should not appeal to these 
        theorems as they would trivialize the proof in this direction). In addition, we are not given any 
        information on what this norm looks like, so we are assuming it is the norm induced by the inner product
        on $V$.

        $\impliedby$ direction

        Let $v\in\text{span}\left({e_1,\dots,e_m}\right)$, and this means that there exists some 
        $a_1,\dots,a_m\in\F$ such that 
        \[
            v = a_1e_1 + \cdots + a_me_m = \sum_{k=1}^ma_ke_k.
        \]
        Now, we substitute this equation into our norm statement and simplify.
        \begin{align*}
            \norm{v}{}^2 &= \ip{v,v} \\
            &= \ip{\sum_{i=1}^ma_ie_i,\sum_{j=1}^ma_je_j} \\
            &= \sum_{i=1}^m\ip{a_ie_i, \sum_{i=j}^ma_je_j} \\
            &= \sum_{i=1}^ma_i\ip{e_i, \sum_{i=j}^ma_je_j} \\
            &= \sum_{i=1}^m\sum_{j=1}^ma_i\ip{e_i,a_je_j} \\
            &= \sum_{i=1}^m\sum_{j=1}^ma_i\overline{a}_j\ip{e_i,e_j} 
        \end{align*}
        However, note that since $e_1,\dots,e_m$ is an orthonormal list of vectors in $V$, we know that for
        all $i,j$
        \[
            \ip{e_i,e_j} = 
            \begin{cases}
                1 & \text{ if }i=j \\
                0 & \text{ otherwise}
            \end{cases}
        \]
        So our double sum can be rewritten as
        \[
            \sum_{i=1}^ma_i\overline{a}_i\ip{e_i,e_i} = \sum_{i=1}^m\abs{a_i}^2
        \]
        which is the exact form we want! Thus, we have shown
        \[
            v\in\text{span}\left({e_1,\dots,e_m}\right) \implies \norm{v}{}^2 = |\ip{v,e_1}|^2 + \cdots + |\ip{v,e_m}|^2
        \]

        Now, let's show the $\implies$ direction. We will do this through contradiction IE assume that
        \[
            \norm{v}{}^2 = |\ip{v,e_1}|^2 + \cdots + |\ip{v,e_m}|^2
        \]
        and $v\not\in\spanS{e_1,\dots,e_m}$.

        Since $v\not\in\spanS{e_1,\dots,e_m}$ but $v\in V$, we first extend $(e_1,\dots,e_m)$ to an orthonormal
        basis of $V$. Denote this basis $(e_1,\dots,e_m,e_{m+1},\dots,e_n)$. We want to create such a basis 
        so that we can show a contradiction using the norm statement. This means that there exists some 
        $a_1,\dots,a_m,a_{m+1},\dots,a_n\in\F$ where at least one of $a_{m+1},\dots,a_n$ non-zero. Let $j$
        be the first index such that $j\geq m+1$ and $a_j\neq 0$ (this will make some later inequalities easier to
        deal with).

        Now, similarly as above, we can write $v$ in terms of a linear combination of this basis. See that
        \[
            v = a_1e_1 + \cdots + a_ne_n = \sum_{k=1}^na_ke_k.
        \] 
        Similarly as above, we rewrite our norm statement and simplify.
        \begin{align*}
            \norm{v}{}^2 &= \ip{v,v} \\
            &= \ip{\sum_{i=1}^na_ie_i,\sum_{j=1}^na_je_j} \\
            &= \sum_{i=1}^n\ip{a_ie_i, \sum_{i=j}^na_je_j} \\
            &= \sum_{i=1}^na_i\ip{e_i, \sum_{i=j}^na_je_j} \\
            &= \sum_{i=1}^n\sum_{j=1}^na_i\ip{e_i,a_je_j} \\
            &= \sum_{i=1}^n\sum_{j=1}^na_i\overline{a}_j\ip{e_i,e_j} \\
            &= \sum_{i=1}^n\abs{a_i}^2\\
            &\geq\sum_{i=1}^m\abs{a_i}^2 + \abs{a_j} \\
            &> \sum_{i=1}^m\abs{a_i}^2 \\
            &= \norm{v}{}^2
        \end{align*}
        So we have that $\norm{v}{}^2 > \norm{v}{}^2$ which is not possible. Thus we have shown that
        \[
            v\in\text{span}\left({e_1,\dots,e_m}\right) \impliedby \norm{v}{}^2 = |\ip{v,e_1}|^2 + \cdots + |\ip{v,e_m}|^2
        \]

        So, in all we have shown that 
        \[
            v\in\text{span}\left({e_1,\dots,e_m}\right) \iff \norm{v}{}^2 = |\ip{v,e_1}|^2 + \cdots + |\ip{v,e_m}|^2
        \] 
        as desired.


        \secName{Solution}
        
        First, we prove the $\impliedby$ direction. IE assume that $v\in\spanS{e_1,\dots,e_m}$. This means that 
        there exists some 
        $a_1,\dots,a_m\in\F$ such that 
        \[
            v = a_1e_1 + \cdots + a_me_m = \sum_{k=1}^ma_ke_k.
        \]
        Next, we compute the norm of $v$. We assume the norm here is the induced norm based on the inner product
        on $V$.
        \begin{align*}
            \norm{v}{}^2 &= \ip{v,v} \\
            &= \ip{\sum_{i=1}^ma_ie_i,\sum_{j=1}^ma_je_j} \\
            &= \sum_{i=1}^m\ip{a_ie_i, \sum_{i=j}^ma_je_j} \\
            &= \sum_{i=1}^ma_i\ip{e_i, \sum_{i=j}^ma_je_j} \\
            &= \sum_{i=1}^m\sum_{j=1}^ma_i\ip{e_i,a_je_j} \\
            &= \sum_{i=1}^m\sum_{j=1}^ma_i\overline{a}_j\ip{e_i,e_j} 
        \end{align*}
        However, note that since $e_1,\dots,e_m$ is an orthonormal list of vectors in $V$, we know that for
        all $i,j$
        \[
            \ip{e_i,e_j} = 
            \begin{cases}
                1 & \text{ if }i=j \\
                0 & \text{ otherwise}
            \end{cases}
        \]
        So our double sum can be rewritten as
        \[
            \sum_{i=1}^ma_i\overline{a}_i\ip{e_i,e_i} = \sum_{i=1}^m\abs{a_i}^2
        \]
        So, we have shown that
        \[
            v\in\text{span}\left({e_1,\dots,e_m}\right) \implies \norm{v}{}^2 = |\ip{v,e_1}|^2 + \cdots + |\ip{v,e_m}|^2
        \]
        as desired.

        Next, we show the $\implies$ direction. We will do this by contradiction. So, assume that
        \[
            \norm{v}{}^2 = |\ip{v,e_1}|^2 + \cdots + |\ip{v,e_m}|^2
        \]
        and that $v\not\in\spanS{e_1,\dots,e_m}$.

        Extend $e_1,\dots,e_m$ to be an orthonormal list of vectors in $V$. We know that we are adding at least
        one term as if $v\in V$ and $v\not\in\spanS{e_1,\dots,e_m}$ then there must be at least one more 
        linearly independent vector in $V$ as otherwise this would be impossible to do. Denote this extended
        basis as $(e_1,\dots,e_m,e_{m+1},\dots,e_n)$. 

        Since this is a basis of $V$, there exist $a_1,\dots,a_n\in\F$ such that 
        \[
            v = \sum_{k=1}^n a_kv_k
        \]

        In addition, we know at least one of $a_{m+1},\dots,a_n$ are non-zero as we are assuming 
        $v\not\in\spanS{e_1,\dots,e_m}$. For ease of notation, let $j$ be the first index such that $a_j\neq0$
        and $j\geq m+1$.

        Similarly as above, we rewrite our norm statement and simplify.
        \begin{align*}
            \norm{v}{}^2 &= \ip{v,v} \\
            &= \ip{\sum_{i=1}^na_ie_i,\sum_{j=1}^na_je_j} \\
            &= \sum_{i=1}^n\ip{a_ie_i, \sum_{i=j}^na_je_j} \\
            &= \sum_{i=1}^na_i\ip{e_i, \sum_{i=j}^na_je_j} \\
            &= \sum_{i=1}^n\sum_{j=1}^na_i\ip{e_i,a_je_j} \\
            &= \sum_{i=1}^n\sum_{j=1}^na_i\overline{a}_j\ip{e_i,e_j} \\
            &= \sum_{i=1}^n\abs{a_i}^2\\
            &\geq\sum_{i=1}^m\abs{a_i}^2 + \abs{a_j} \\
            &> \sum_{i=1}^m\abs{a_i}^2 \\
            &= \norm{v}{}^2
        \end{align*}
        So we have that $\norm{v}{}^2 > \norm{v}{}^2$ which is our contradiction. Thus we have shown that
        \[
            v\in\text{span}\left({e_1,\dots,e_m}\right) \impliedby \norm{v}{}^2 = |\ip{v,e_1}|^2 + \cdots + |\ip{v,e_m}|^2
        \]

        So, in all we have shown that 
        \[
            v\in\text{span}\left({e_1,\dots,e_m}\right) \iff \norm{v}{}^2 = |\ip{v,e_1}|^2 + \cdots + |\ip{v,e_m}|^2
        \] 
        as desired.
        
    \end{solution}
\end{questions}

%\section{Other Problems}
\end{document}
