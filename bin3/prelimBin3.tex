% Hacky way of allowing us to compile this with and without answers using the makefile in this folder
\ifdefined\isanswers
    \documentclass[answers]{exam}
\else
    \documentclass{exam}
\fi
% Use mystyle file. This is a congolmeration of things I've used over the years. Most is stolen from either previous professors or stack overflow. It may or may not be cited. Assume I didn't come up with any of it myself.
\usepackage{mystyle}
\renewcommand{\L}[1]{\mathcal{L}\left(#1\right)}
\newcommand{\ip}[1]{\left\langle#1\right\rangle}
\newcommand{\M}[1]{\mathcal{M}\left(#1\right)}

\title{Bin 3 Problems}
\date{\today}
\author{}

\begin{document}
\maketitle
This document holds problems that fit into the Bin 4 according to the prelim syllabus. Or are approached 
in a way most compatible with Bin 3

\tableofcontents

\section{Prelim Problems}
\begin{questions}
    \question Let $A$ be a Hermitian $n\times n$ complex matrix . Show that if 
    \question Let $T$ be a positive operator on a complex inner product space $V$ and $S$ be an operator on $V$
    such that $ST = -TS$. Show that $ST=TS=0$.
    \begin{solution}
        \secName{Intuition} Since in this problem, we are not given much information about $S,T$ aside from the
        fact that $T$ is positive and $S$ and $T$ nearly commute. So we should look at eigenvalues and 
        eigenvectors! Since $T$ is positive, we know it has non-negative eigenvalues and is self
        adjoint from Axler 7.35. So from the Complex spectral theorem, we have that $T$ is diagonalizable in an
        orthonormal basis consisting of eigenvectors of $T$. Let $\lambda$ be an eigenvale of $T$ with associated
        eigenvector $v$. This means $Tv = \lambda v$. However we also know that
        \begin{align*}
            \lambda v = Tv &\iff S\lambda v = STv\\
            &\iff \lambda Sv = STv \\
            &\iff \lambda Sv = -TSv\\
            &\iff -\lambda Sv = TSv
        \end{align*}
        So we have two cases. The first case is that $Sv = 0$. If this is the case, then $STv = -TSv = 0$.
        However assume that $Sv\neq 0$. This means that $Sv$ is an eigenvector of $T$ associated with eigenvalue
        $-\lambda$. Since all eigenvalues of $T$ are non-negative we have that both
        \begin{align*}
            \lambda &\geq 0 \\
            -\lambda&\geq 0
        \end{align*}
        This means that $\lambda = 0$. Then, $STv = -TSv = 0$.

        Since we know that for all eigenvectors of $T$, we have that $TSv = STv = 0$, we know this also holds 
        for a basis consisting of eigenvectors of $T$, so $TS=ST=0$ as desired.


        \secName{Solution} Since $T$ is positive, by Axler 7.35 it is both self-adjoint and has non-negative
        eigenvalues. Since $T$ is self adjoint, we know that there exists an orthonormal basis of $V$ consisting
        of eigenvectors of $T$. Denote this basis $v_1,\dots,v_n$. Now, let $(\lambda,v)$ be an arbitrary 
        eigenpair of $T$. IE $Tv = \lambda v$. We see that
        \begin{align*}
            \lambda v = Tv &\iff S\lambda v = STv \\
            &\iff \lambda Sv = STv \\
            &\iff \lambda Sv = -TSv \\
            &\iff TSv = -\lambda Sv
        \end{align*}

        From here, we have $2$ cases for $Sv$, the first being that $Sv = \vec{0}$. If this is the case then
        $STv = TSv = \vec{0}$. Otherwise, $Sv\neq 0$. This means that $Sv$ is an eigenvector of $T$ associated
        with $-\lambda$. However, since $T$ has non-negative eigenvalues we know
        \begin{align*}
            \lambda &\geq 0 \\
            -\lambda&\geq 0
        \end{align*}
        So, $\lambda = 0$. In this case we know $STv = TSv = 0$. Since we have that for all eigenvectors of $T$
        denoted $v$,
        $STv = TSv = 0$. We also know this is the case for $(v_1,\dots,v_n)$ (our basis of eigenvectors of $T$). 
        Since for all $k=1,\dots,n$, $STv_k = TSv_k = 0$, $ST=TS=0$ as desired.
    \end{solution}
\end{questions}

\section{Axler Problems}
\begin{questions}
\end{questions}

\section{Other Problems}
\begin{questions}
\end{questions}
\end{document}
