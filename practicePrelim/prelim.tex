% Hacky way of allowing us to compile this with and without answers using the makefile in this folder
\ifdefined\isanswers
    \documentclass[answers]{exam}
\else
    \documentclass{exam}
\fi
% Use mystyle file. This is a congolmeration of things I've used over the years. Most is stolen from either previous professors or stack overflow. It may or may not be cited. Assume I didn't come up with any of it myself.
\usepackage{mystyle}

\newcommand{\spanV}[1]{\text{span}\left(#1\right)}
\renewcommand{\L}[1]{\mathcal{L}\left(#1\right)}
\newcommand{\ip}[1]{\left\langle#1\right\rangle}
\newcommand{\M}[1]{\mathcal{M}\left(#1\right)}

\title{Practice Prelim}
\date{\today}
\author{}

\begin{document}
\maketitle

Assume that $\F$ is a field of elements either real or complex (ie any proofs using $\F$ should hold for either
real or complex numbers)

\section*{Section 1 Problems}

\begin{questions}
    % Axler 2.A Exercise 11
    \question Let $V$ be a vector space over a field $\F$. Suppose $v_1,\dots,v_m$ is linearly independent in $V$ and $w\in V$. Show that $v_1,\dots,v_m,w$ is
    linearly independent if and only if 
    \[
        w\not\in\spanV{v_1,\dots,v_m}
    \]
    % Axler 6.C Exercise 3
    \question Let $V$ be an inner product space over a field $\F$. Suppose that $U$ is a subspace of $V$ and 
    $u_1,\dots,u_m$ forms a basis of $U$. Furthermore, suppose that $u_1,\dots,u_m,w_1,\dots,w_n$ forms a basis
    of $V$. Prove that if the Gram-Schmidt Procedure is applied to the basis of $V$ above, producing a list
    $e_1,\dots,e_m,f_1,\dots,f_n$, then $e_1,\dots,e_m$ forms an orthonormal basis of $U$ and $f_1,\dots,f_n$ is
    an orthonormal basis of $U^\perp$

    \textbf{\underline{Note:}} If you are unfamiliar with the Gram-Schmidt Procedure, see Theorem 6.31 in Axler. 
    Familiarity with this theorem is expected, but you will probably not be expected to apply it. (However as 
    always ask the committee as they are the only ones qualified to make such statements with exactness).
    % 2021 August prelim problem 3
    \question For each of the following $4$ statements, give either a counterexample or a reason why it is true.
    \begin{parts}
        \part For every real matrix $A$ there is a real matrix $B$ with $B^{-1}AB$ diagonal.
        \part For every symmetric real matrix $A$, there is a real matrix $B$ with $B^{-1}AB$ diagonal.
        \part For every complex matrix $A$ there is a complex matrix $B$ with $B^{-1}AB$ diagonal.
        \part For every symmetric complex matrix $A$ there is a complex matrix $B$ with $B^{-1}AB$ diagonal.
    \end{parts}
    % Axler  7.B Exercise 5
    \question Suppose $V$ is a real inner product space and $T\in\L{V}$. Prove that $T$ is self-adjoint if and
    only if all pairs of eigenvectors corresponding to distinct eigenvalues of $T$ are orthogonal and 
    \[
        V = E(\lambda_1,T)\oplus \cdots \oplus E(\lambda_m,T),
    \]
    where $\lambda_1,\dots,\lambda_m$ denote the distinct eigenvalues of $T$.

    \textbf{\underline{Note:}} This is very similar to prooving $(a)\implies (d)$ in Theorem 5.41. So, you 
    cannot use this theorem!
\end{questions}
\newpage
\section*{Section 2 Problems}
Pick 2 of the following \\
\begin{questions}
\setcounter{question}{4}
    % 2021 January prelim problem 5
    \question Define $\R^{n\times n}$ to be the space of all real $n-\text{by}-n$ matrices, suppose $S\in\R^{n\times n}$,
    and define the linear mapping
    \[
        \mathcal{T}:\R^{n\times n}\rightarrow\R^{n\times n}, \qquad \mathcal{T}:P\mapsto PS + SP
    \]
    \begin{parts}
        \part Prove that if $\lambda$ is an eigenvalue of $S$, $u$ is the corresponding eigenvector, and 
        $u\in\nullS{\mathcal{T}P}$, then $Pu$ is also an eigenvector of $S$ with eigenvalue $-\lambda$.
        \part Prove that if $S$ is symmetric positive definite, then the mapping $\mathcal{T}$ is injective.
    \end{parts}
    % 2021 August prelim problem 6
    \question Let $A$ be a real $3\times 3$ symmetric matrix, whose eigenvalues are $\lambda_1,\lambda_2, 
    \text{ and } \lambda_3.$ Prove the following:
    \begin{parts}
        \part If the trace of $A$, is not an eigenvalue of $A$, then $(\lambda_1+\lambda_2)(\lambda_2+\lambda_3)
        (\lambda_1+\lambda_3)\neq 0$
        \part If $(\lambda_1+\lambda_2)(\lambda_2+\lambda_3)(\lambda_1+\lambda_3)\neq 0$, then the map 
        $T:S\rightarrow S$ is an isomorphism, where $T(W) = AW + WA$ and $S$ is the space of $3\times 3$ real skew-symmetric matrices
        (if $W^\top = -W$, then $W$ is called skew-symmetric).
    \end{parts}
    % 2022 August prelim problem 7.
    \question Let $T$ be a linear operator on a four dimensional complex vector space that satisfies the 
    polynomial equation $P(T) = T^4 + 2T^3 - 2T - I = 0$ where $I$ is the identity operator on $V$. Suppose
    that $\left|\text{trace}\left(T\right)\right| = 2$ and that $\dim\range\left(T + I\right) = 2$. Give
    a Jordan canonical form of $T$.
    % 2022 February prelim problem 8. This one is hard. I would skip this one if I was taking the test
    \question Let $A$ be an $n-\text{by}-n$ matrix with complex entries. Prove that A is the sum of two nonsingular matrices. 
\end{questions}
\end{document}
